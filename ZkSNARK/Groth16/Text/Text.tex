\documentclass{article}
\usepackage[utf8]{inputenc}
\usepackage{amsmath} % Required for some math elements
\usepackage{amsfonts}
\usepackage{amsthm} % for proof pkg
\usepackage{amssymb}
\usepackage{verbatim}
\usepackage{hyperref}
\usepackage{comment}
\usepackage{color}
\usepackage{parskip} % Remove paragraph indentation
\usepackage[margin=1in]{geometry}
\usepackage[inline]{enumitem}
\usepackage{mathtools}
\usepackage{multirow}
\usepackage{listings}

\newtheorem{lemma}{Lemma}
\newtheorem{claim}{Claim}
\newtheorem{theorem}{Theorem}

\theoremstyle{definition}
\newtheorem{definition}{Definition}[section]
\newtheorem{fact}{Fact}[section]
\theoremstyle{remark}
\newtheorem*{remark}{Remark}

\newcommand{\anote}[1]{{\color{magenta} [AM: #1]}}
\newcommand{\Gen}{\mathsf{Gen}}
\newcommand{\Enc}{\mathsf{Enc}}
\newcommand{\Dec}{\mathsf{Dec}}
\newcommand{\pk}{\mathit{pk}}
\newcommand{\sk}{\mathit{sk}}
\newcommand{\bbZ}{\mathbb{Z}}
\newcommand{\bit}{\{0,1\}}
\newcommand{\la}{\leftarrow}
\newcommand{\ninN}{{n \in \mathbf{N}}}
\newcommand{\cF}{\mathcal{F}}
\newcommand{\cG}{\mathcal{G}}
\newcommand{\RF}{\mathsf{RF}}
\newcommand{\Half}{\frac{1}{2}}
\newcommand{\F}{\mathbb{F}}
\newcommand{\Adv}{\mathcal{A}}
\newcommand{\Ext}{\mathcal{E}}

\newcommand{\ignore}[1]{{}}

\newcommand{\samples}{\overset{\$}{\leftarrow}}
\newcommand{\hash}{\ensuremath{\mathcal{H}}}
\newcommand{\doubleplus}{+\kern-1.3ex+\kern0.8ex}
\newcommand{\mdoubleplus}{\ensuremath{\mathbin{+\mkern-10mu+}}}

\title{Formalising Groth16 in Lean 4}
\author{Daniel Rogozin, for Yatima Inc}
\date{\today\footnote{This document may be updated frequently.}}

\begin{document}

\maketitle

In this document, we describe the Groth16 soundness formalisation in Lean 4.
The text contains the protocol description as well as some comments to its implementation.

\section{Preliminary definitions}

We have a fixed finite field $F$, and $F[X]$ stands for the ring of polynomials over $F$ as usual. The corresponding listing:

\begin{lstlisting}
variable {F : Type u} [field : Field F]
\end{lstlisting}

In Groth16, we have random values $\alpha, \beta, \gamma, \delta \in F$ that we introduce separately as an inductive data type:
\begin{lstlisting}
inductive Vars : Type
  | alpha : Vars
  | beta : Vars
  | gamma : Vars
  | delta : Vars
\end{lstlisting}

We also introduce the following parameters:

\begin{itemize}
\item $n_{stmt} \in \mathbb{N}$ --- the statement size;
\item $n_{wit} \in \mathbb{N}$ --- the witness size;
\item $n_{var} \in \mathbb{N}$ --- the number of variables.
\end{itemize}

In Lean 4, we introduce those parameters as variables in the following way:

\begin{lstlisting}
variable {n_stmt n_wit n_var : Nat}
\end{lstlisting}

We also define several finite collections of polynomials:

\begin{itemize}
\item $u_{stmt} = \{ f_{i} \in F[X] \: | \: i < n_{stmt} \}$
\item $u_{wit} = \{ f_{i} \in F[X] \: | \: i < n_{wit} \}$
\item $v_{stmt} = \{ f_{i} \in F[X] \: | \: i < n_{stmt} \}$
\item $v_{wit} = \{ f_{i} \in F[X] \: | \: i < n_{wit} \}$
\item $w_{stmt} = \{ f_{i} \in F[X] \: | \: i < n_{stmt} \}$
\item $w_{wit} = \{ f_{i} \in F[X] \: | \: i < n_{wit} \}$
\end{itemize}

We introduce those collections in Lean 4 as variables as well:

\begin{lstlisting}
variable {u_stmt : Fin n_stmt -> F[X]}
variable {u_wit : Fin n_wit -> F[X]}
variable {v_stmt : Fin n_stmt -> F[X]}
variable {v_wit : Fin n_wit -> F[X]}
variable {w_stmt : Fin n_stmt -> F[X]}
variable {w_wit : Fin n_wit -> F[X]}
\end{lstlisting}

Let $(r_i)_{i < n_{wit}}$ be a collection of elements of $F$ (that is, each $r_i \in F$) parametrised by elements of $n_{wit}$. Define a polynomial $t \in F[X]$ as:
\begin{center}
$t = \prod \limits_{i \in n_{wit}} (x - r_i)$.
\end{center}

Crearly, these $r_i$'s are roots of $t$. The definition in Lean 4:
\begin{lstlisting}
variable (r : Fin n_wit -> F)

def t : F[X] := Pi i in finRange n_wit,
  (x : F[X]) - Polynomial.c (r i)
\end{lstlisting}

\section{Properties of $t$}

The polynomial $t$ has the following properties:

\begin{lemma}
$ $

\begin{enumerate}
\item $deg(t) = n_{wit}$;
\item $t$ is monic, that is, its leading coefficient is equal to $0$;
\item If $n_{wit} > 0$, then $deg(t) > 0$.
\end{enumerate}
\end{lemma}

We formalise these statements as follows (but we drop proofs):
\begin{lstlisting}
lemma nat_degree_t : (t r).natDegree = n_wit
lemma monic_t : Polynomial.Monic (t r)
lemma degree_t_pos (hm : 0 < n_wit) : 0 < (t r).degree
\end{lstlisting}

\end{document}
