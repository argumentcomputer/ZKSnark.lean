\documentclass{article}
\usepackage[T1]{fontenc}
\usepackage[utf8]{inputenc}
\usepackage{amsmath} % Required for some math elements
\usepackage{amsfonts}
\usepackage{amsthm} % for proof pkg
\usepackage{amssymb}
\usepackage{verbatim}
\usepackage{hyperref}
\usepackage{comment}
\usepackage{color}
\usepackage{parskip} % Remove paragraph indentation
\usepackage[margin=1in]{geometry}
\usepackage[inline]{enumitem}
\usepackage{mathtools}
\usepackage{multirow}
\usepackage{listings}

\usepackage{color}
\definecolor{keywordcolor}{rgb}{0.7, 0.1, 0.1}   % red
\definecolor{tacticcolor}{rgb}{0.0, 0.1, 0.6}    % blue
\definecolor{commentcolor}{rgb}{0.4, 0.4, 0.4}   % grey
\definecolor{symbolcolor}{rgb}{0.0, 0.1, 0.6}    % blue
\definecolor{sortcolor}{rgb}{0.1, 0.5, 0.1}      % green
\definecolor{attributecolor}{rgb}{0.7, 0.1, 0.1} % red

\def\lstlanguagefiles{lstlean.tex}
% set default language
\lstset{language=lean}

\newtheorem{lemma}{Lemma}
\newtheorem{claim}{Claim}
\newtheorem{theorem}{Theorem}

\theoremstyle{definition}
\newtheorem{definition}{Definition}[section]
\newtheorem{fact}{Fact}[section]
\theoremstyle{remark}
\newtheorem*{remark}{Remark}

\newcommand{\anote}[1]{{\color{magenta} [AM: #1]}}
\newcommand{\Gen}{\mathsf{Gen}}
\newcommand{\Enc}{\mathsf{Enc}}
\newcommand{\Dec}{\mathsf{Dec}}
\newcommand{\pk}{\mathit{pk}}
\newcommand{\sk}{\mathit{sk}}
\newcommand{\bbZ}{\mathbb{Z}}
\newcommand{\bit}{\{0,1\}}
\newcommand{\la}{\leftarrow}
\newcommand{\ninN}{{n \in \mathbf{N}}}
\newcommand{\cF}{\mathcal{F}}
\newcommand{\cG}{\mathcal{G}}
\newcommand{\RF}{\mathsf{RF}}
\newcommand{\Half}{\frac{1}{2}}
\newcommand{\F}{\mathbb{F}}
\newcommand{\Adv}{\mathcal{A}}
\newcommand{\Ext}{\mathcal{E}}

\newcommand{\ignore}[1]{{}}

\newcommand{\samples}{\overset{\$}{\leftarrow}}
\newcommand{\hash}{\ensuremath{\mathcal{H}}}
\newcommand{\doubleplus}{+\kern-1.3ex+\kern0.8ex}
\newcommand{\mdoubleplus}{\ensuremath{\mathbin{+\mkern-10mu+}}}

\title{Formalising Groth16 in Lean 4}
\author{Daniel Rogozin, for Yatima Inc}
\date{\today\footnote{This document may be updated frequently.}}

\begin{document}

\maketitle

\section{Introduction}

In this document, we describe \href{https://github.com/yatima-inc/ZKSnark.lean/blob/main/ZkSNARK/Groth16/KnowledgeSoundness.lean}{the Groth16 soundness formalisation} in Lean 4.
The text contains the protocol description as well as some comments to its implementation.

Groth16 is a kind of ZK-SNARK protocol introduced in~\cite{groth2016size}. The latter means that:

\begin{itemize}
\item It is \emph{zero-knowledge}. 
  In other words, a prover has only a particular piece of information. 
  A prover proves that they know some value but the value remains disclosed.
\item It is \emph{non-interactive}, that is, there is no back and forth interaction 
between a prover and a verifier. In particular, this allows reusing secret parameters, 
having efficient performance, and verifying statements at a small cost one.
\end{itemize}

Protocols of this kind have the core characteristics such as:
\begin{itemize}
\item \emph{Soundness}, i.e., if a statement does not hold, then the prover cannot convince the verifier.
\item \emph{Completeness}, i.e., the verifier is convinced whenever a statement is true.
\item \emph{Zero-knowledge}, i.e., the only thing is needed is the truth of a statement.
\end{itemize}

Generally, non-interactive zero-knowledge proofs relies on the \emph{common reference string} model, 
that is, a model where a public string is generated in a trusted way and all parties have an access to it.
One can think of a common reference string as a public set of parameters.

Let us describe the commond scheme that non-interactive zero-knowledge protocols obey, see~\cite{petkus2019and} and~\cite{bitansky2012extractable} to have more details. Before that, we need a bit of terminology.

Let $p \in F[X]$ be a polynomial, a prover is going to convince a verifier that they know $p$. In turn, knowing $p$ means that a prover knows some of its roots. As it is well-known, 
any polynomial might be decomposed as follows whenever it has roots\footnote{Since fields we consider are finite and they are not algebraically closed}:
\begin{equation}
p(x) = \prod_{i = 0}^{\deg(p)} (x - a_i)
\end{equation}
for some $a_i$, $i < \deg(p)$.

Assume that a prover has some values $\{ r_i \: | \: i < n\}$ where each $r_i \in F$ for some $n \leq \deg(p)$.
A prover wants to convince a verifier that $p(r_i)$ for each $a_i$ from that set.

If there $a_i$'s are really roots of $p$, then the polynomial $p$ can be rewritten as:
\begin{equation}
p(x) = \left( \prod_{i = 0}^n (x - r_i) \right) \cdot h(x)
\end{equation}
for some $h \in F[X]$.

Denote $\prod_{i = 0}^n (x - r_i)$ as $t(x)$. We shall call $t(x)$ further a \emph{target polynomial}. So, a verifier accepts only if a target polynomial $t$ divides $p$, in particular, that means that all those $r_i$'s are roots of $p$.

The next notion we need is a square span program (see~\cite{danezis2014square}) for verification of which the target polynomial is used. Originally, it has been introduced as a simpler version of quadratic span programs for an alternative characterisation of the $\operatorname{NP}$ complexity class.

A square span program is defined rigorously as:
\begin{definition}
Let $F$ be a field and $m$ a natural number. A \emph{square span program} $Q$ over $F$ is a collection of polynomials $t_0, \dots, t_m \in F[X]$ and a target polynomial $t$ such that:
\begin{center}
$\forall i \leq m \: \deg(t_i) \leq \deg(t)$
\end{center}

Let $1 \leq l \leq m$, then a square span program $Q$ accepts a tuple $(a_1, \dots, a_l) \in F^l$ iff

\begin{center}
$\exists a_{l + 1}, \dots, a_m \in F \: \left( t(x) \: | \: {\left( t_0(x) + \sum \limits_{i = 1}^m a_i t_i(x) \right)}^2 - 1 \right)$
\end{center}
\end{definition}

Square span programs are $\operatorname{NP}$-complete and it is proved by reducing them to the Boolean satisfability problem. 
We focus on their application in non-interactive zero-knowledge arguments.

TODO: describe that in more detail

Now we discuss specific aspects of Groth16 in addition the aforedescribed general ZK-SNARK scheme.
Groth16 has been introduced as an NIZK-argument with a proof consisting of only 3 group elements.
Here the verifier merely computes approximately the same number of exponentiations as the statement size
and checks a single pairing product equation that has only 3 pairings. 
In particular, this makes the proof quite easy to verify. 
Our formalisation also includes the Type III pairing which is one of the most efficient ones.

\section{Preliminary definitions}

We have a fixed finite field $F$, and $F[X]$ stands for the polynomial ring over $F$ as usual. The corresponding listing written in Lean:

\begin{lstlisting}
variable {F : Type u} [field : Field F]
\end{lstlisting}

In Groth16, we have random values $\alpha, \beta, \gamma, \delta \in F$ that we introduce separately as an inductive data type:
\begin{lstlisting}
inductive Vars : Type
  | alpha : Vars
  | beta : Vars
  | gamma : Vars
  | delta : Vars
\end{lstlisting}
TODO: explain semantic roles of $\alpha, \beta, \gamma, \delta$.

We also introduce the following parameters:

\begin{itemize}
\item $n_{stmt} \in \mathbb{N}$ --- the statement size;
\item $n_{wit} \in \mathbb{N}$ --- the witness size;
\item $n_{var} \in \mathbb{N}$ --- the number of variables.
\end{itemize}
where $n_{wit}$ is the degree of the target polynomial

In Lean 4, we introduce those parameters as variables in the following way:

\begin{lstlisting}
variable {n_stmt n_wit n_var : Nat}
\end{lstlisting}

We also define the following finite collections of polynomials from the square span program:

\begin{itemize}
\item $u_{stmt} = \{ f_{i} \in F[X] \: | \: i < n_{stmt} \}$
\item $u_{wit} = \{ f_{i} \in F[X] \: | \: i < n_{wit} \}$
\item $v_{stmt} = \{ f_{i} \in F[X] \: | \: i < n_{stmt} \}$
\item $v_{wit} = \{ f_{i} \in F[X] \: | \: i < n_{wit} \}$
\item $w_{stmt} = \{ f_{i} \in F[X] \: | \: i < n_{stmt} \}$
\item $w_{wit} = \{ f_{i} \in F[X] \: | \: i < n_{wit} \}$
\end{itemize}

We introduce those collections in Lean 4 as variables as well:

\begin{lstlisting}
variable {u_stmt : Finₓ n_stmt → F[X]}
variable {u_wit : Finₓ n_wit → F[X]}
variable {v_stmt : Finₓ n_stmt → F[X]}
variable {v_wit : Finₓ n_wit → F[X]}
variable {w_stmt : Finₓ n_stmt → F[X]}
variable {w_wit : Finₓ n_wit → F[X]}
\end{lstlisting}

Let $(r_i)_{i < n_{wit}}$ be a collection of elements of $F$ (that is, each $r_i \in F$) parametrised with $\{0, \dots, n_{wit} \}$.
Define the target polynomial $t \in F[X]$ of degree $n_{wit}$ as:
\begin{center}
$t = \prod \limits_{i = 0}^{n_{wit}} (x - r_i)$.
\end{center}

Crearly, these $r_i$'s are roots of $t$. The definition in Lean 4:
\begin{lstlisting}
variable (r : Finₓ n_wit → F)
def t : F[X] := ∏ i in finRange n_wit, (x : F[X]) - Polynomial.c (r i)
\end{lstlisting}

We think of the collection \verb"r" as roots of the polynomial \verb"t" as it can be observed from the definition.
We use divisibility of $t$ to verify the square span program condition.

The polynomial $t$ has the following self-evident properties:

\begin{lemma}
$ $

\begin{enumerate}
\item $\deg(t) = n_{wit}$;
\item $t$ is monic, that is, its leading coefficient is equal to $1$;
\item If $n_{wit} > 0$, then $\deg(t) > 0$.
\end{enumerate}
\end{lemma}

The aforemention properties is a common place for the target polynomial from zero-knowledge protocols. 
We formalise these statements as follows (but we skip proofs):
\begin{lstlisting}
lemma nat_degree_t : (t r).natDegree = n_wit
lemma monic_t : Polynomial.Monic (t r)
lemma degree_t_pos (hm : 0 < n_wit) : 0 < (t r).degree
\end{lstlisting}

Let $\{ a_{{wit}_i} \: | \: i < n_{wit}\}$ and $\{ a_{{stmt}_i} \: | \: i < n_{stmt} \}$ be collections of elements of $F$.
A stamenent witness polynomial pair is a pair of single variable polynomials $(F_{{wit}_{sv}}, F_{{stmt}_{sv}})$
such that $F_{{wit}_{sv}}, F_{{stmt}_{sv}} \in F[X]$ and
\begin{itemize}
\item $F_{{wit}_{sv}} = \sum \limits_{i = 0}^{n_{wit}} a_{{wit}_i} u_{{wit}_{i}}(x)$
\item $F_{{stmt}_{sv}} = \sum \limits_{i = 0}^{n_{stmt}} a_{{stmt}_i} u_{{stmt}_{i}}(x)$
\end{itemize}

Their Lean 4 counterparts:
\begin{lstlisting}
def V_wit_sv (a_wit : Finₓ n_wit → F) : F[X] :=
  ∑ i in finRange n_wit, a_wit i • u_wit i

def V_stmt_sv (a_stmt : Finₓ n_stmt → F) : F[X] :=
  ∑ i in finRange n_stmt, a_stmt i • u_stmt i
\end{lstlisting}

Define the polynomial $sat$ as:
\begin{equation}
sat = \sum \limits_{i = 0}^{n_{stmt}} a_{{stmt}_i} v_{{stmt}_i}(x) + \sum \limits_{i = 0}^{n_{wit}} a_{{wit}_i} v_{{wit}_i}(x) - \sum \limits_{i = 0}^{n_{stmt}} a_{{stmt}_i} w_{{stmt}_i}(x) + \sum \limits_{i = 0}^{n_{wit}} a_{{wit}_i} w_{{wit}_i}(x)
\end{equation}
A pair $(F_{{wit}_{sv}}, F_{{stmt}_{sv}})$ satisfies \emph{the square span program},
if the remainder of division of $sat$ by $t$ is equal to $0$.
This requirement is common for ZK-SNARK protocols and the square span program in general as we discussed in the introduction.

The Lean 4 analogue of the property defined above:
\begin{lstlisting}
def satisfying (a_stmt : Finₓ n_stmt → F) (a_wit : Finₓ n_wit → F) :=
  (((∑ i in finRange n_stmt, a_stmt i • u_stmt i)
    + ∑ i in finRange n_wit, a_wit i • u_wit i) *
  ((∑ i in finRange n_stmt, a_stmt i • v_stmt i)
    + ∑ i in finRange n_wit, a_wit i • v_wit i) -
  ((∑ i in finRange n_stmt, a_stmt i • w_stmt i)
    + ∑ i in finRange n_wit, a_wit i • w_wit i) : F[X]) %ₘ (t r) = 0
\end{lstlisting}

\section{Common reference string elements}

Assume we interpreted $\alpha$, $\beta$, $\gamma$, and $\delta$ somehow with elements of $F$, 
say $crs_{\alpha}$, $crs_{\beta}$, $crs_{\gamma}$, and $crs_{\delta}$.
In Lean 4, we define those elements as higher-order function with an interpretation function 
$f : Vars \to F$ as an argument.

\begin{lstlisting}
def crs_α  (f : Vars → F) : F := f Vars.α
def crs_β (f : Vars → F) : F := f Vars.β
def crs_γ (f : Vars → F) : F := f Vars.γ
def crs_δ (f : Vars → F) : F := f Vars.δ
\end{lstlisting}

TODO: explain semantic roles of the CRS polynomials
In addition to those four elements of $F$ we have a collection of degrees for $a \in F$:
\begin{center}
$\{ a^i \: | \: i < n_{var} \}$
\end{center}
formalised as:
\begin{lstlisting}
def crs_powers_of_x (i : Finₓ n_var) (a : F) : F := (a)^(i : ℕ)
\end{lstlisting}

We also introduce collections $crs_l$, $crs_m$, and $crs_n$ for $a \in F$:
\begin{multline}
crs_l = \frac{((f(\beta) / f(\gamma)) \cdot (u_{{stmt}_i})(a)) + ((f(\alpha) / f(\gamma)) \cdot (v_{{stmt}_i})(a)) + w_{{stmt}_i}(a)}{f(\gamma)} \\ \text{for $i < n_{stmt}$}
\end{multline}

\begin{multline}
crs_l = \frac{((f(\beta) / f(\delta)) \cdot (u_{{wit}_i})(a)) + ((f(\alpha) / f(\delta)) \cdot (v_{{wit}_i})(a)) + w_{{wit}_i}(a)}{f(\delta)} \\ \text{for $i < n_{wit}$}
\end{multline}

\begin{equation}
crs_l = \frac{a^i \cdot t(a)}{f(\delta)}, \text{for $i < n_{var}$}
\end{equation}

Their Lean 4 versions:

\begin{lstlisting}
def crs_l (i : Finₓ n_stmt) (f : Vars → F) (a : F) : F :=
  ((f Vars.β / f Vars.γ) * (u_stmt i).eval (a) +
  (f Vars.α / f Vars.γ) * (v_stmt i).eval (a) +
  (w_stmt i).eval (a)) / f Vars.γ

def crs_m (i : Finₓ n_wit) (f : Vars → F) (a : F) : F :=
  ((f Vars.β / f Vars.δ) * (u_wit i).eval (a) +
  (f Vars.α / f Vars.δ) * (v_wit i).eval (a) +
  (w_wit i).eval (a)) / f Vars.δ

def crs_n (i : Finₓ (n_var - 1)) (f : Vars → F) (a : F) : F :=
  ((a)^(i : ℕ)) * (t r).eval a / f Vars.δ
\end{lstlisting}

Assume we have fixed elements of a field $A_{\alpha}$, $A_{\beta}$, $A_{\gamma}$, $A_{\delta}$, $B_{\alpha}$, $B_{\beta}$, $B_{\gamma}$, $B_{\delta}$, $C_{\alpha}$, $C_{\beta}$, $C_{\gamma}$, $C_{\delta} \in F$.

We also have indexed collections:
\begin{itemize}
\item $\{ A_x \in F \: | \: x < n_{var} \}$
\item $\{ B_x \in F \: | \: x < n_{var} \}$
\item $\{ C_x \in F \: | \: x < n_{var} \}$
\item $\{ A_l \in F \: | \: l < n_{stmt} \}$
\item $\{ B_l \in F \: | \: l < n_{stmt} \}$
\item $\{ C_l \in F \: | \: l < n_{stmt} \}$
\item $\{ A_m \in F \: | \: m < n_{wit} \}$
\item $\{ B_m \in F \: | \: m < n_{wit} \}$
\item $\{ C_m \in F \: | \: m < n_{wit} \}$
\item $\{ A_h \in F \: | \: h < n_{var - 1} \}$
\item $\{ B_h \in F \: | \: h < n_{var - 1} \}$
\item $\{ C_h \in F \: | \: h < n_{var - 1} \}$
\end{itemize}
TODO: explain

\begin{lstlisting}
variable { A_α A_β A_γ A_δ B_α B_β B_γ B_δ C_α C_β C_γ C_δ  : F }
variable { A_x B_x C_x : Finₓ n_var → F }
variable { A_l B_l C_l : Finₓ n_stmt → F }
variable { A_m B_m C_m : Finₓ n_wit → F }
variable { A_h B_h C_h : Finₓ (n_var - 1) → F }
\end{lstlisting}

The adversary's proof representation is defined as the following three sums, for $x \in F$:

\begin{multline}
A = A_{\alpha} \cdot crs_{\alpha} + A_{\beta} \cdot crs_{\beta} + A_{\gamma} \cdot crs_{\gamma} + A_{\delta} \cdot crs_{\delta} + \\
    + \sum \limits_{i = 0}^{n_{var}} A_{x_i} * x^i + \sum \limits_{i = 0}^{n_{stmt}} A_{l_i} * crs_l(x) + \\
    + \sum \limits_{i = 0}^{n_{wit}} A_{m_i} * crs_m(x) + \sum \limits_{i = 0}^{n_{var} - 1} A_{h_i} * crs_n(x)
\end{multline}

\begin{multline}
B = B_{\alpha} \cdot crs_{\alpha} + B_{\beta} \cdot crs_{\beta} + B_{\gamma} \cdot crs_{\gamma} + B_{\delta} \cdot crs_{\delta} + \\
    + \sum \limits_{i = 0}^{n_{var}} B_{x_i} * x^i + \sum \limits_{i = 0}^{n_{stmt}} B_{l_i} * crs_l(x) + \\
    + \sum \limits_{i = 0}^{n_{wit}} B_{m_i} * crs_m(x) + \sum \limits_{i = 0}^{n_{var} - 1} B_{h_i} * crs_n(x)
\end{multline}

\begin{multline}
C = C_{\alpha} \cdot crs_{\alpha} + C_{\beta} \cdot crs_{\beta} + C_{\gamma} \cdot crs_{\gamma} + C_{\delta} \cdot crs_{\delta} + \\
    + \sum \limits_{i = 0}^{n_{var}} C_{x_i} * x^i + \sum \limits_{i = 0}^{n_{stmt}} C_{l_i} * crs_l(x) + \\
    + \sum \limits_{i = 0}^{n_{wit}} C_{m_i} * crs_m(x) + \sum \limits_{i = 0}^{n_{var} - 1} C_{h_i} * crs_n(x)
\end{multline}
TODO: explain

Here, we provide the Lean 4 version of $A$ only.

\begin{lstlisting}
def A (f : Vars → F) (x : F) : F :=
  A_α * crs_α F f + A_β * crs_β F f + A_γ * crs_γ F f + A_δ * crs_δ F f +
  ∑ i in (finRange n_var), (A_x i) * (crs_powers_of_x F i x) +
  ∑ i in (finRange n_stmt), (A_l i) * (@crs_l F field n_stmt u_stmt v_stmt w_stmt i f x) +
  ∑ i in (finRange n_wit), (A_m i) * (@crs_m F field n_wit u_wit v_wit w_wit i f x) +
  ∑ i in (finRange (n_var - 1)), (A_h i) * (crs_n F r i f x)
\end{lstlisting}

A proof is called \emph{verified}, if the following equation holds:
\begin{equation}
A \cdot B = crs_{\alpha} \cdot crs_{\beta} + \left(\sum \limits_{i = 0}^{n_{stmt}} a_{{stmt}_i} \cdot crs_{l_i}(x) \right) \cdot crs_{\gamma} + C \cdot crs_{\delta}
\end{equation}

\begin{lstlisting}
def verified (f : Vars -> F) (x : F) (a_stmt : Fin n_stmt -> F ) : Prop :=
  A f x * B f x =
    (crs_alpha F f * crs_beta F f) +
    ((\sum i in finRange n_stmt, (a_stmt i) * @crs_l i f x) *
      (crs_gamma F f) + C f x * (crs_delta F f))
\end{lstlisting}

\section{Modified common reference string elements}

We modify common reference string elements from the previous section as multivariate polynomials.

\subsection{Coefficient lemmas}

TODO: those lemmas are rather technical but it's worth providing a couple of examples.

\section{Formalised soundness}

TODO: describe soundness

\section{Groth16, Type III}

In this section, we describe the Lean 4 formalisation of a Groth16 version called Type III, see \cite{baghery2021another}. It has certain specific moments.

\subsection{Type III pairing}

TODO: motivate Type III

In Type III, polynomials $A$, $B$, $C$ have a slightly more simple form:
\begin{lstlisting}
  def A (f : Vars → F) : F[X] :=
  (Polynomial.c A_α) * crs_α F f + (Polynomial.c A_β) * crs_β F f +
  (Polynomial.c A_δ) * crs_δ F f +
  ∑ i in (finRange n_var), (Polynomial.c (A_x i)) * (crs_powers_of_x F i) +
  ∑ i in (finRange n_stmt), (Polynomial.c (A_l i)) * (@crs_l F field n_stmt u_stmt v_stmt w_stmt i f) +
  ∑ i in (finRange n_wit), (Polynomial.c (A_m i)) * (@crs_m F field n_wit u_wit v_wit w_wit i f) +
  ∑ i in (finRange (n_var-1)), (Polynomial.c (A_h i)) * (crs_n F r i f)
\end{lstlisting}

\begin{lstlisting}
def B (f : Vars → F) : F[X] :=
  (Polynomial.c B_β) * (crs_β F f) + (Polynomial.c B_γ) * (crs_γ F f) +
  (Polynomial.c B_δ) * (crs_δ F f) +
  ∑ i in (finRange n_var), (Polynomial.c (B_x i)) * (crs_powers_of_x F i)
\end{lstlisting}

\begin{lstlisting}
def C (f : Vars → F) : F[X]  :=
  (Polynomial.c C_α) * crs_α F f + (Polynomial.c C_β) * crs_β F f +
  (Polynomial.c C_δ) * crs_δ F f +
  ∑ i in (finRange n_var), (Polynomial.c (C_x i)) * (crs_powers_of_x F i) +
  ∑ i in (finRange n_stmt), (Polynomial.c (C_l i)) * (@crs_l F field n_stmt u_stmt v_stmt w_stmt i f) +
  ∑ i in (finRange n_wit), (Polynomial.c (C_m i)) * (@crs_m F field n_wit u_wit v_wit w_wit i f) +
  ∑ i in (finRange (n_var - 1)), (Polynomial.c (C_h i)) * (crs_n F r i f)
\end{lstlisting}

\bibliographystyle{unsrt}
\bibliography{refs}

\end{document}
