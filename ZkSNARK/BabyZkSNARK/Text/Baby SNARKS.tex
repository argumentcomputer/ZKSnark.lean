\documentclass{article}
\usepackage[utf8]{inputenc}
\usepackage{amsmath} % Required for some math elements 
\usepackage{amsfonts}
\usepackage{amsthm} % for proof pkg
\usepackage{amssymb}
\usepackage{verbatim}
\usepackage{hyperref}
\usepackage{comment}
\usepackage{color}
\usepackage{parskip} % Remove paragraph indentation
\usepackage[margin=1in]{geometry}
\usepackage[inline]{enumitem}
\usepackage{mathtools}
\usepackage{multirow}

\newtheorem{lemma}{Lemma}
\newtheorem{claim}{Claim}
\newtheorem{theorem}{Theorem}
 
\theoremstyle{definition}
\newtheorem{definition}{Definition}[section]
\newtheorem{fact}{Fact}[section]
\theoremstyle{remark}
\newtheorem*{remark}{Remark}

\newcommand{\anote}[1]{{\color{magenta} [AM: #1]}}
\newcommand{\Gen}{\mathsf{Gen}}
\newcommand{\Enc}{\mathsf{Enc}}
\newcommand{\Dec}{\mathsf{Dec}}
\newcommand{\pk}{\mathit{pk}}
\newcommand{\sk}{\mathit{sk}}
\newcommand{\bbZ}{\mathbb{Z}}
\newcommand{\bit}{\{0,1\}}
\newcommand{\la}{\leftarrow}
\newcommand{\ninN}{{n \in \mathbf{N}}}
\newcommand{\cF}{\mathcal{F}}
\newcommand{\cG}{\mathcal{G}}
\newcommand{\RF}{\mathsf{RF}}
\newcommand{\Half}{\frac{1}{2}}
\newcommand{\F}{\mathbb{F}}
\newcommand{\Adv}{\mathcal{A}}
\newcommand{\Ext}{\mathcal{E}}

\newcommand{\ignore}[1]{{}}

\newcommand{\samples}{\overset{\$}{\leftarrow}}
\newcommand{\hash}{\ensuremath{\mathcal{H}}}
\newcommand{\doubleplus}{+\kern-1.3ex+\kern0.8ex}
\newcommand{\mdoubleplus}{\ensuremath{\mathbin{+\mkern-10mu+}}}

\title{Baby SNARKs}
\author{Ashvni Narayanan, for Yatima Inc}
\date{\today\footnote{This document may be updated frequently.}}

\begin{document}

\maketitle

This file describes the implementation of the soundness proof of the Baby SNARKs program. The aim is to explain the code of the proof of soundness in \cite{code}. The mathematics is given in \cite{main}, and the code shall be explained in the same notation.

\section{Setup}
Some notes :
\begin{itemize}
  \item The \texttt{open\_locale big\_operators} command lets us use the local notation for sums ($\sum$) and products ($\prod$), as defined in the file \cite{big_operators}.
  \item By declaring \texttt{universes u}, one assumes that all elements have \texttt{Type u}.
  \item \texttt{parameters} is the same as \texttt{variables}, and is used to declare variables that have scope in a given \texttt{section}. In this case, they are valid throughout the file.
  \item It would help to \texttt{open polynomial} (open the namespace \texttt{polynomial}) at the beginning of the file, one then does not need to add the prefix to each lemma that is called from that namespace.
\end{itemize}

We have as variables \texttt{F}, which is a field (although it is mentioned that this is the finite field parameter of the SNARK, the finiteness is nowhere stated or used). We also 
have the natural number variables \texttt{m}, \texttt{n\_stmt} and \texttt{n\_wit}. These are $m$, $l$ and $n - l$ in \cite{main}. $n$ is defined to be the sum of \texttt{n\_stmt} and \texttt{n\_wit}. 

The collection of polynomials $u_0, u_1, \cdots u_{l - 1}$ are defined \href{https://github.com/BoltonBailey/formal-snarks-project/blob/7fd9cd122f5887f88f6a706b4f2a68a7153c7381/src/snarks/babysnark/knowledge_soundness.lean#L59}{here}. 
The author defines it in terms of a function \texttt{u\_stmt}, which takes an element of $\mathbb{Z}/l \mathbb{Z}$ and returns a polynomial with $F$-coefficients. Note that \texttt{fin n\_stmt} is nothing but the set of natural numbers 
up to $l$, or equivalently, $\mathbb{Z}/l \mathbb{Z}$. \texttt{u\_wit} is defined similarly to denote the polynomials $u_l, u_{l + 1}, \cdots u_{n - 1}$. The roots of the polynomial $t$ are defined in the same fashion, with 
\texttt{r i} denoting $r_i$, for $0 \le i \le m - 1$. 

The polynomial $t$ is then defined as $t = \prod_{i = 0}^{m - 1} (X - r_i)$ \href{https://github.com/BoltonBailey/formal-snarks-project/blob/7fd9cd122f5887f88f6a706b4f2a68a7153c7381/src/snarks/babysnark/knowledge_soundness.lean#L66}{here}. 
\texttt{polynomial.X} denotes $X$ as a polynomial in $F[X]$, and \texttt{polynomial.C (r i)} denotes the constant polynomial $r_i$.

\section{Properties of $t$}
The lemma \href{https://github.com/BoltonBailey/formal-snarks-project/blob/7fd9cd122f5887f88f6a706b4f2a68a7153c7381/src/snarks/babysnark/knowledge_soundness.lean#L71}{\texttt{nat\_degree\_t}} says : 
\theoremstyle{Lemma}
\begin{lemma}[]
  The degree of $t$ is $m$.  
\end{lemma}

\texttt{nat\_degree} returns the degree of the polynomial as a natural number. This differs from \texttt{polynomial.degree} only when the polynomial is zero. 
The proof follows simply by noting that the degree of the product of the polynomials $\prod_{i = 0}^{m - 1} (X - r_i)$ is the sum of the degrees of $X - r_i$ (\texttt{nat\_degree\_prod}), as long as each of these are nonzero (\texttt{X\_sub\_C\_ne\_zero}).

The lemma \href{https://github.com/BoltonBailey/formal-snarks-project/blob/7fd9cd122f5887f88f6a706b4f2a68a7153c7381/src/snarks/babysnark/knowledge_soundness.lean#L82}{\texttt{monic\_t}} then says :
\theoremstyle{lemma}
\begin{lemma} \label{monic_t}
  The polynomial $t$ is monic.
\end{lemma}
The proof follows from the fact that a product of monic polynomials is monic (\texttt{monic\_prod\_of\_monic}), and that each $(X - r_i)$ is monic (\texttt{monic\_X\_sub\_C}).

The next lemma \href{https://github.com/BoltonBailey/formal-snarks-project/blob/7fd9cd122f5887f88f6a706b4f2a68a7153c7381/src/snarks/babysnark/knowledge_soundness.lean#L91}{\texttt{degree\_t\_pos}} tells us :
\theoremstyle{lemma}
\begin{lemma} \label{t_pos}
  If $0 < m$, then the degree of $t$ is positive.
\end{lemma}
Note that this lemma uses \texttt{degree} instead of \texttt{nat\_degree}. As a result, we must prove that $m$ is nonzero implies $t$ being nonzero, in which case \texttt{nat\_degree} and \texttt{degree} coincide. 

Before getting into the proof, let us first understand the reason for the distinction between \texttt{nat\_degree} and \texttt{degree}. Lean uses the inductive type \texttt{option}. Basically, given \texttt{A}, 
\texttt{option A} comprises of \texttt{none} (the undefined element) and \texttt{some a} for all elements \texttt{a} of \texttt{A}. The function \texttt{option.get\_or\_else a} returns \texttt{b} when given 
\texttt{some b} and \texttt{a} when given \texttt{none}. Given a polynomial \texttt{p}, \texttt{degree p} returns \texttt{some} of the supremum of all numbers $n$ such that $X^n$ has 
a nonzero coefficient in $p$. When $p = 0$, this returns the supremum of the empty set, $\bot$, which is the same as \texttt{none}. \texttt{nat\_degree} is then defined to be \texttt{(degree p).get\_or\_else 0} : 
if \texttt{degree p} is $\bot$, it returns 0, and \texttt{(degree p)} otherwise.

We first show that it suffices to prove that \texttt{degree t = some m}. This follows easily from the fact that \texttt{0 < some m} implies \texttt{0 < m} (\texttt{with\_bot.some\_lt\_some}).
The proof is then by induction on \texttt{degree t}. 
If \texttt{degree t = none}, then a contradiction is derived, since we then have that \texttt{some m = none}, which then implies \texttt{m < m}, which is false. In the other case, we have that \texttt{degree t = some val} 
for some value \texttt{val}. Then by the definition of \texttt{option.get\_or\_else}, we get that \texttt{m = val}, and the proof follows simply from Lemma 1.

\section{Some definitions}
One of the fundamental concepts used in this proof is that single variable polynomials can also be thought of as multi-variable polynomials. In this section, we give the mechanism to translate between the two, as well as 
define the polynomials $V_w$, $V_s$, $B_w$, $V$, $H$ etc, sometimes separately as both single and multivariable polynomials. 

Let us first understand the conversion between single and multivariable polynomials. The author defines \texttt{vars} to be an inductive type used to index 3-variable polynomials (we shall assume the variables are $X$, $Y$ 
and $Z$ throughout). They then define \href{https://github.com/BoltonBailey/formal-snarks-project/blob/7fd9cd122f5887f88f6a706b4f2a68a7153c7381/src/snarks/babysnark/knowledge_soundness.lean#L139}{\texttt{singlify}} to 
convert 3-variable polynomials to a single variable one : \texttt{singlify} replaces the coefficients $Y$ and $Z$ with 1 and leaves $X$ as it is.

On the other side, \href{https://github.com/BoltonBailey/formal-snarks-project/blob/7fd9cd122f5887f88f6a706b4f2a68a7153c7381/src/snarks/babysnark/knowledge_soundness.lean#L145}{\texttt{X\_poly}}, \texttt{Y\_poly} and 
\texttt{Z\_poly} are $X$, $Y$ and $Z$ thought of as elements of $F[X, Y, Z]$.

Any finitely supported function taking values in $\mathbb{N}$ can be thought of as a polynomial. As an example, any finitely supported function 
$m : \texttt{vars} \to \mathbb{N}$ can be thought of as a monomial of the form $X^{m(\texttt{vars.X})} Y^{m(\texttt{vars.Y})} Z^{m(\texttt{vars.Z})}$. Then, 
given a multivariable polynomial $p(X, Y, Z)$, \texttt{mv\_polynomial.coeff p m} denotes the coefficient of $p$ at the monomial $m$.

Given a ring $S$, a multivariable polynomial $p := \sum_{m \in \sigma} a_{\sigma} m$ ($\sigma$ is a set of monomials) with coefficients in $R$, 
a function $f : R \to S$, and a function $g : \sigma \to S$, \texttt{mv\_polynomial.eval\_2 f g p :=} $\sum_{m \in \sigma} f(a_{\sigma}) g(m)$ returns the evaluation of $p$ with respect to $f$ and $g$. 
In our case, we get, for $p := \sum_{i, j, k = 0}^{1} a_{i, j ,k} X^i Y^j Z^k$, we get \texttt{mv\_polynomial.eval\_2 polynomial.C singlify p :=} $\sum_{i, j, k = 0}^{1} a_{i, j ,k} X^i$ as an element of $F[X]$.

We now give the definitions of various single and multivariable polynomials :
\begin{itemize}
  \item \href{https://github.com/BoltonBailey/formal-snarks-project/blob/7fd9cd122f5887f88f6a706b4f2a68a7153c7381/src/snarks/babysnark/knowledge_soundness.lean#L120}{\texttt{V\_wit\_sv}} : Given $a_w = (a_l, \cdots, a_{n - 1})$, returns $V_w(X) := \sum_{i = l}^{n - 1} a_w (i) u_i (X)$ as an element of $F[X]$.
  \item \href{https://github.com/BoltonBailey/formal-snarks-project/blob/7fd9cd122f5887f88f6a706b4f2a68a7153c7381/src/snarks/babysnark/knowledge_soundness.lean#L124}{\texttt{V\_stmt\_sv}} : Given $a_s = (a_0, \cdots, a_{l - 1})$, returns $V_s(X) := \sum_{i = 0}^{l - 1} a_s (i) u_i (X)$ as an element of $F[X]$.
  \item \href{https://github.com/BoltonBailey/formal-snarks-project/blob/7fd9cd122f5887f88f6a706b4f2a68a7153c7381/src/snarks/babysnark/knowledge_soundness.lean#L153}{\texttt{V\_stmt\_mv}} : Given $a_s = (a_0, \cdots, a_{l - 1})$, returns $V_s(X, Y, Z) := V_s(X)$ as an element of $F[X, Y, Z]$.
  \item \href{https://github.com/BoltonBailey/formal-snarks-project/blob/7fd9cd122f5887f88f6a706b4f2a68a7153c7381/src/snarks/babysnark/knowledge_soundness.lean#L150}{\texttt{t\_mv}} : Returns $t(X, Y, Z) := t(X)$ as an element of $F[X, Y, Z]$.
  \item \href{https://github.com/BoltonBailey/formal-snarks-project/blob/7fd9cd122f5887f88f6a706b4f2a68a7153c7381/src/snarks/babysnark/knowledge_soundness.lean#L166}{\texttt{crs\_powers\_of\_t}} : Given $i \in \{ 0, \cdots, m - 1 \}$, returns $X^i$ as an element of $F[X, Y, Z]$.
  \item \href{https://github.com/BoltonBailey/formal-snarks-project/blob/7fd9cd122f5887f88f6a706b4f2a68a7153c7381/src/snarks/babysnark/knowledge_soundness.lean#L167}{\texttt{crs\_g}} : Returns $Z$ as an element of $F[X, Y, Z]$.
  \item \href{https://github.com/BoltonBailey/formal-snarks-project/blob/7fd9cd122f5887f88f6a706b4f2a68a7153c7381/src/snarks/babysnark/knowledge_soundness.lean#L168}{\texttt{crs\_gb}} : Returns $Z Y$ as an element of $F[X, Y, Z]$.
  \item \href{https://github.com/BoltonBailey/formal-snarks-project/blob/7fd9cd122f5887f88f6a706b4f2a68a7153c7381/src/snarks/babysnark/knowledge_soundness.lean#L169}{\texttt{crs\_b\_ssps}} : Given $i \in \{ l, \cdots, n - 1 \}$, returns $Y u_i(X)$ as an element of $F[X, Y, Z]$.
\end{itemize}

We also have the variables \texttt{b}, \texttt{v} and \texttt{h} which are functions/strings of length $m$, $\mathbb{Z}/m \mathbb{Z} \to F$ representing $(b_i)_{i = 0}^{m - 1}$, $(v_i)_{i = 0}^{m - 1}$ and $(h_i)_{i = 0}^{m - 1}$ respectively; 
\texttt{b'}, \texttt{v'} and \texttt{h'} which are functions/strings of length $n - l$, $\mathbb{Z}/(n - l) \mathbb{Z} \to F$ representing $(b'_i)_{i = l}^{n - l - 1}$, $(v'_i)_{i = l}^{n - l - 1}$ and $(h'_i)_{i = l}^{n - l - 1}$ respectively; and 
\texttt{b\_g v\_g h\_g b\_gb v\_gb h\_gb}, which are elements of $F$, representing $b_{\gamma}, v_{\gamma}, h_{\gamma}, b_{\gamma \beta}, v_{\gamma \beta}, h_{\gamma \beta}$ respectively.

We can now define the main polynomials used :
\begin{itemize}
  \item \href{https://github.com/BoltonBailey/formal-snarks-project/blob/7fd9cd122f5887f88f6a706b4f2a68a7153c7381/src/snarks/babysnark/knowledge_soundness.lean#L177}{\texttt{B\_wit}} : Returns $B_w := \sum_{i = 0}^{m - 1} b_i X^i + b_{\gamma}Z + b_{\gamma \beta}YZ + \sum_{i = l}^{n - 1} b'_{i} Y u_{i}(X)$ as an element of $F[X, Y, Z]$
  \item \href{https://github.com/BoltonBailey/formal-snarks-project/blob/7fd9cd122f5887f88f6a706b4f2a68a7153c7381/src/snarks/babysnark/knowledge_soundness.lean#L186}{\texttt{V\_wit}} : Returns $V_w := \sum_{i = 0}^{m - 1} v_i X^i + v_{\gamma}Z + v_{\gamma \beta}YZ + \sum_{i = l}^{n - 1} v'_{i} Y u_{i}(X)$ as an element of $F[X, Y, Z]$
  \item \href{https://github.com/BoltonBailey/formal-snarks-project/blob/7fd9cd122f5887f88f6a706b4f2a68a7153c7381/src/snarks/babysnark/knowledge_soundness.lean#L195}{\texttt{H}} : Returns $H := \sum_{i = 0}^{m - 1} h_i X^i + h_{\gamma}Z + h_{\gamma \beta}YZ + \sum_{i = l}^{n - 1} h'_{i} Y u_{i}(X)$ as an element of $F[X, Y, Z]$
  \item \href{https://github.com/BoltonBailey/formal-snarks-project/blob/7fd9cd122f5887f88f6a706b4f2a68a7153c7381/src/snarks/babysnark/knowledge_soundness.lean#L206}{\texttt{V}} : Given $a_s = (a_0, \cdots, a_{l - 1})$, returns $V := V_w + V_s$ as an element of $F[X, Y, Z]$
\end{itemize}

The above information is encapsulated in the following table :
\begin{center}
  \begin{tabular}{ c c c c }
    \textbf{Lean} & \textbf{Text} & \textbf{Description} & \textbf{Type} \\
    \texttt{X\_poly} & $X$ & $X$ & $F[X, Y, Z]$ \\
    \texttt{Y\_poly} & $Y$ & $Y$ & $F[X, Y, Z]$ \\
    \texttt{Z\_poly} & $Z$ & $Z$ & $F[X, Y, Z]$ \\
    \texttt{V\_wit\_sv} & $V_w(X)$ & $\sum_{i = l}^{n - 1} a_w (i) u_i (X)$ & $F[X]$ \\
    \texttt{V\_stmt\_sv} & $V_s(X)$ & $\sum_{i = 0}^{l - 1} a_s (i) u_i (X)$ & $F[X]$ \\
    \texttt{V\_stmt\_mv} & $V_s(X)$ & $\sum_{i = 0}^{l - 1} a_s (i) u_i (X)$ & $F[X, Y, Z]$ \\
    \texttt{t\_mv} & $t(X)$ & $t(X)$ & $F[X, Y, Z]$ \\
    \texttt{crs\_powers\_of\_t i} & $X^i$ & $X^i$ & $F[X, Y, Z]$ \\
    \texttt{crs\_g} & $Z$ & $Z$ & $F[X, Y, Z]$ \\
    \texttt{crs\_gb} & $ZY$ & $ZY$ & $F[X, Y, Z]$ \\
    \texttt{crs\_b\_ssps i} & $Y u_i (X)$ & $F[X, Y, Z]$ \\
    \texttt{b} & $(b_i)_{i = 0}^{m - 1}$ & $(b_i)_{i = 0}^{m - 1}$ & $\mathbb{Z}/m \mathbb{Z} \to F$ \\
    \texttt{v} & $(v_i)_{i = 0}^{m - 1}$ & $(v_i)_{i = 0}^{m - 1}$ & $\mathbb{Z}/m \mathbb{Z} \to F$ \\
    \texttt{h} & $(h_i)_{i = 0}^{m - 1}$ & $(h_i)_{i = 0}^{m - 1}$ & $\mathbb{Z}/m \mathbb{Z} \to F$ \\
    \texttt{b'} & $(b'_i)_{i = l}^{n - l - 1}$ & $(b'_i)_{i = l}^{n - l - 1}$ & $\mathbb{Z}/(n - l) \mathbb{Z} \to F$ \\
    \texttt{v'} & $(v'_i)_{i = l}^{n - l - 1}$ & $(v'_i)_{i = l}^{n - l - 1}$ & $\mathbb{Z}/(n - l) \mathbb{Z} \to F$ \\
    \texttt{h'} & $(h'_i)_{i = l}^{n - l - 1}$ & $(h'_i)_{i = l}^{n - l - 1}$ & $\mathbb{Z}/(n - l) \mathbb{Z} \to F$ \\
    \texttt{b\_g} & $b_{\gamma}$ & $b_{\gamma}$ & $F$ \\
    \texttt{v\_g} & $v_{\gamma}$ & $v_{\gamma}$ & $F$ \\
    \texttt{h\_g} & $h_{\gamma}$ & $h_{\gamma}$ & $F$ \\
    \texttt{b\_gb} & $b_{\gamma \beta}$ & $b_{\gamma \beta}$ & $F$ \\
    \texttt{v\_gb} & $v_{\gamma \beta}$ & $v_{\gamma \beta}$ & $F$ \\
    \texttt{h\_gb} & $h_{\gamma \beta}$ & $F$ \\
    \texttt{B\_wit} & $B_w$ & $\sum_{i = 0}^{m - 1} b_i X^i + b_{\gamma}Z + b_{\gamma \beta}YZ + \sum_{i = l}^{n - 1} b'_{i} Y u_{i}(X)$ & $F[X, Y, Z]$ \\
    \texttt{V\_wit} & $V_w$ & $\sum_{i = 0}^{m - 1} v_i X^i + v_{\gamma}Z + v_{\gamma \beta}YZ + \sum_{i = l}^{n - 1} v'_{i} Y u_{i}(X)$ & $F[X, Y, Z]$ \\
    \texttt{H} & $H$ & $\sum_{i = 0}^{m - 1} h_i X^i + h_{\gamma}Z + h_{\gamma \beta}YZ + \sum_{i = l}^{n - 1} h'_{i} Y u_{i}(X)$ & $F[X, Y, Z]$ \\
    \texttt{V} & $V$ & $V_s + V_w$ & $F[X, Y, Z]$ \\
  \end{tabular}
\end{center}

Finally, we say that the pair $(a_i)_{i = 0}^{l - 1}$ and $(a_i)_{i = l}^{n - 1}$ is \texttt{satisfying} if 
$$ \sum_{i = 0}^{l - 1} a_i u_i(X) + \sum_{i = l}^{n - 1} a_i u_i (X) \equiv 1 \texttt{mod } t $$
that is, on dividing the above polynomial by $t$, the remainder obtained is 1. The significance of looking at these sums 
separately is that the witness information is only available to the prover, not the verifier.

\section{Supporting lemmas}
In this section we state some lemmas that shall assist us in the proof of the final theorem.

The lemma \href{https://github.com/BoltonBailey/formal-snarks-project/blob/7fd9cd122f5887f88f6a706b4f2a68a7153c7381/src/snarks/babysnark/knowledge_soundness.lean#L158}{\texttt{my\_multivariable\_to\_single\_variable}} 
states that :
\theoremstyle{lemma}
\begin{lemma} \label{eval2}
  Given a polynomial $p$ in $F[X]$, we have \texttt{(mv\_polynomial.eval\_2 polynomial.C singlify (eval\_2 mv\_polynomial.C X\_poly p)) = p}
\end{lemma}
This gives tells us that converting a single variable polynomial to a multivariable polynomial and reducing it to a single variable polynomial returns the original polynomial. 
The proof remains the same if \texttt{singlify} is replaced with an arbitrary function. Thus we use the lemma 
\href{https://github.com/BoltonBailey/formal-snarks-project/blob/7fd9cd122f5887f88f6a706b4f2a68a7153c7381/src/general_lemmas/polynomial_mv_sv_cast.lean#L14}{\texttt{multivariable\_to\_single\_variable}}. The proof of this lemma 
follows simply by unfolding the definitions, using properties of monomials, and the fact that any polynomial $p$ can be written as $\sum_{n} coeff_{p}(X^n) X^n$, along with some \texttt{simp} lemmas.

The following lemma \href{https://github.com/BoltonBailey/formal-snarks-project/blob/7fd9cd122f5887f88f6a706b4f2a68a7153c7381/src/snarks/babysnark/knowledge_soundness.lean#L211}{\texttt{eq\_helper}} 
is used in \texttt{h2\_1} :
\theoremstyle{lemma}
\begin{lemma} 
    Given natural numbers $x$ and $n$, $x = j \iff x = j \vee (x = 0 \wedge j = 0) $
\end{lemma}

This lemma seems obvious, however, it is quite useful to state beforehand, so it can be used directly in the next lemma.
The proof is simple, we split the goal into two statements and get two goals : $x = j \to x = j \vee (x = 0 \wedge j = 0)$ and 
$x = j \vee (x = 0 \wedge j = 0) \to x = j$. The first implication is trivial. We must split the second implication into 2 cases :
$x = j \to x = j$ and $x = 0 \wedge j = 0 \to x = j$. Both implications are trivial. 

The next lemma, \href{https://github.com/BoltonBailey/formal-snarks-project/blob/7fd9cd122f5887f88f6a706b4f2a68a7153c7381/src/snarks/babysnark/knowledge_soundness.lean#L223}{\texttt{h2\_1}} states that :
\theoremstyle{lemma} \label{h2_1}
\begin{lemma}
    $\forall 0 \le i < m$, the coefficient of $X^{i}$ in $B_{w}$ (or \texttt{B\_wit}) is $b_i$.
\end{lemma}

The lemma follows by tracking quotients, unfolding various definitions, removing coercions and applying the lemmas \texttt{finsupp.single\_eq\_single\_iff}, \texttt{eq\_helper} and \texttt{fin.eq\_iff\_veq}. This is done by applying the tactics 
\texttt{simp} and \texttt{unfold\_coes}. For a full list of lemmas that \texttt{simp} uses, one can apply \texttt{squeeze\_simp}.

Following a similar proof as above, the lemma \href{https://github.com/BoltonBailey/formal-snarks-project/blob/7fd9cd122f5887f88f6a706b4f2a68a7153c7381/src/snarks/babysnark/knowledge_soundness.lean#L244}{\texttt{h3\_1}} is proved : 
\theoremstyle{lemma} \label{h3_1}
\begin{lemma}
    The coefficient of $Z$ in $B_{w}$ (or \texttt{B\_wit}) is $b_{\gamma}$.
\end{lemma}

In fact, a single \texttt{simp} proves this, with an addition of \texttt{finsupp.single\_eq\_single\_iff}.

The lemma \href{https://github.com/BoltonBailey/formal-snarks-project/blob/7fd9cd122f5887f88f6a706b4f2a68a7153c7381/src/snarks/babysnark/knowledge_soundness.lean#L261}{\texttt{h4\_1}} says :
\theoremstyle{lemma}
\begin{lemma} \label{h4_1}
    Suppose that, $\forall 0 \le i < m, b_{i} = 0$. Then, $b_i X^i = 0$. Equivalently, the function defined as $f(i) := b_i \cdot X^i$ is the same as the zero function.
\end{lemma}

The lemma is stated in the function form. Here, $\cdot$ represents scalar multiplication of $F$ on $F[X]$.
The proof uses the tactic \texttt{ext}, which says that functions $f$ and $g$ are equal if and only if $\forall x, f(x) = g(x)$. The conclusion follows from 
using the hypothesis and applying \texttt{zero\_smul}.

The lemma \href{https://github.com/BoltonBailey/formal-snarks-project/blob/7fd9cd122f5887f88f6a706b4f2a68a7153c7381/src/snarks/babysnark/knowledge_soundness.lean#L272}{\texttt{h5\_1}} says :
\theoremstyle{lemma} \label{h5_1}
\begin{lemma}
    $b_{\gamma \beta} \cdot ZY = Y (b_{\gamma \beta} \cdot Z)$
\end{lemma}

The lemma uses the fact \texttt{mv\_polynomial.smul\_eq\_C\_mul}, which says that scalar multiplication of a polynomial by a constant in $F$ is the same as multiplication of 
the polynomial by the constant polynomial, that is $b \cdot p(X) = b(X) * p(X)$, where $b \in F$ and a polynomial $p(X) \in F[X]$. The tactic \texttt{ring} then finishes the proof 
by using associativity and commutativity of multiplication. One can check what \texttt{ring} does by looking at \texttt{show\_term\{ring\}}.

The lemma \href{https://github.com/BoltonBailey/formal-snarks-project/blob/7fd9cd122f5887f88f6a706b4f2a68a7153c7381/src/snarks/babysnark/knowledge_soundness.lean#L279}{\texttt{h6\_2}} says : 
\theoremstyle{lemma}
\begin{lemma} \label{h6_2}
  The coefficient of $Z^2$ in $H t + 1$ is 0.
\end{lemma}

The coefficient of $Z^2$ in $H t + 1$ is precisely the coefficient of $Z^2$ in $H t$, which is the same as $\sum_{i= 0}^2 coeff_{H}(Z^i) coeff_{t}(Z^{2 - i})$.
We know that $coeff_{t}(Z^i)$ is 0 for every $i$, which concludes the proof.

The lemma \href{https://github.com/BoltonBailey/formal-snarks-project/blob/7fd9cd122f5887f88f6a706b4f2a68a7153c7381/src/snarks/babysnark/knowledge_soundness.lean#L311}{\texttt{h6\_3}} says : 
\theoremstyle{lemma}
\begin{lemma} \label{h6_3}
  Given $(a_i)_{i = 0}^{l - 1}$, the coefficient of $Z^2$ in $(b_{\gamma \beta} \cdot Z + \sum_{i = 0}^{l - 1} a_i u_i (X) + \sum_{j = l}^{n - 1} b'_i u_i (X))^2$ is $b_{\gamma \beta}^2$.
\end{lemma}

The mathematical proof follows by looking at the coefficients. The code relies on first computing the power. This is done by looking at $z^2 = z * z$. One then uses 
\texttt{mv\_polynomial.coeff\_mul} to write out the coefficients in terms of sums over the \texttt{antidiagonal}, which is the same as using the binomial theorem. Given a 
finitely supported function $s$ taking values in the natural numbers, \texttt{antidiagonal s} is the set $\{ (m, n) | m + n = s \}$. Given an element $s$ of type $S$, 
the function \texttt{finsupp.single s n} should be interpreted as taking value $n$ at $s$ and $0$ at all other elements of \texttt{vars}. Then, it is easy to see that the lemma 
\href{https://github.com/BoltonBailey/formal-snarks-project/blob/7fd9cd122f5887f88f6a706b4f2a68a7153c7381/src/general_lemmas/single_antidiagonal.lean#L230}{\texttt{single\_2\_antidiagonal}} follows :
\theoremstyle{lemma}
\begin{lemma}
  Given an element $s$ of a random type $S$, 
  \newline \texttt{antidiagonal (finsupp.single s 2) = 
\{
  (finsupp.single s 0, finsupp.single s 2), 
  (finsupp.single s 1, finsupp.single s 1), 
  (finsupp.single s 2, finsupp.single s 0), 
\} }

\end{lemma}
We then use \texttt{finset.sum\_insert} (writing the sum over a union of sets as an addition of the sums over each of the sets), \texttt{finsupp.single\_eq\_single\_iff} 
(\texttt{finsupp.single a c} = \texttt{finsupp.single b d} $\iff a = b \wedge c = d \vee c = d = 0$) and \texttt{finset.sum\_singleton} (evaluating sums over singleton sets), along with some \texttt{simp} lemmas.
%%We then use \texttt{single\_2\_antidiagonal} to convert the set of elements in the sum with a coefficient of \texttt{antidiagonal ()} and then showing that the irrelevant coefficients 

\section{Main theorem}
Let us first make a definition. We define the \href{https://github.com/BoltonBailey/formal-snarks-project/blob/7fd9cd122f5887f88f6a706b4f2a68a7153c7381/src/snarks/babysnark/knowledge_soundness.lean#L346}{\texttt{extractor}} 
to be the function $b' : \mathbb{Z}/(n - l) \mathbb{Z} \to F$.

We are now ready to prove that the Baby SNARK protocol satisfies the property of knowledge soundness. This means that, if an adversary produces polynomials 
$B(X, Y, Z), V(X, Y, Z), H(X, Y, Z)$ which satisfy 
\begin{equation}
  \label{1} B_w = YV_w
\end{equation}
\begin{equation}
  \label{2} Ht = V^2 - 1  
\end{equation}
then one can extract a suitable \texttt{satisfying} witness that the adversary must have knowledge of, which is precisely $b'$.

We now look at Case 1 in the paper \cite{main}, that is, when the above equations hold for all $X, Y, Z$. This is called 
\href{https://github.com/BoltonBailey/formal-snarks-project/blob/7fd9cd122f5887f88f6a706b4f2a68a7153c7381/src/snarks/babysnark/knowledge_soundness.lean#L351}{\texttt{case\_1}} : 
\theoremstyle{theorem}
\begin{theorem}
  Given $(a_i)_{i = 0}^{l - 1}$, $m > 0$, and $B_w, V_w, H$ satisfy the equations \eqref{1} and \eqref{2} then $a_s$ and the \texttt{extractor} $b'$ are \texttt{satisfying}.
\end{theorem}

The proof is done combining 12 sub-lemmas.

\subsection{Sub-lemmas}
The lemma \href{https://github.com/BoltonBailey/formal-snarks-project/blob/7fd9cd122f5887f88f6a706b4f2a68a7153c7381/src/snarks/babysnark/knowledge_soundness.lean#L362}{\texttt{h1}} states that :
\theoremstyle{lemma}
\begin{lemma}
  Given a monomial $m$ not having a $Y$-term (thought of as a finitely supported function from \texttt{vars} to $\mathbb{N}$), the coefficient of $m$ in $B_w$ is 0.
\end{lemma}
This follows directly from \eqref{1} and \texttt{mul\_var\_no\_constant}, which says that given a monomial $m$ with no $s$-term and a multivariable polynomial $a$, the coefficient of $m$ in $a * s$ is 0.

The lemma \href{https://github.com/BoltonBailey/formal-snarks-project/blob/7fd9cd122f5887f88f6a706b4f2a68a7153c7381/src/snarks/babysnark/knowledge_soundness.lean#L366}{h2} states that :
\theoremstyle{lemma}
\begin{lemma} \label{h2}
  $\forall 0 \le i \le m - 1$, $b_i = 0$.
\end{lemma}
Given $0 \le i \le m - 1$, using \ref{h2_1}, we must prove that the coefficient of $X^i$ in $B_w$ is 0. One then uses \refeq{1}, 
\href{https://github.com/BoltonBailey/formal-snarks-project/blob/7fd9cd122f5887f88f6a706b4f2a68a7153c7381/src/general_lemmas/mv_divisability.lean#L193}{\texttt{mul\_var\_no\_constant}} and 
\texttt{finsupp.single\_apply} (\texttt{(finsupp.single a b) a'} is b if \texttt{a = a'}, otherwise it is \texttt{0}), along with some \texttt{simp} lemmas.

The lemma \href{https://github.com/BoltonBailey/formal-snarks-project/blob/7fd9cd122f5887f88f6a706b4f2a68a7153c7381/src/snarks/babysnark/knowledge_soundness.lean#L366}{h3} states that :
\theoremstyle{lemma}
\begin{lemma} \label{h3}
  $b_{\gamma} = 0$
\end{lemma}
First, one uses \ref{h3_1}, which reduces to showing that the coefficient of $Z$ in $B_w$ is 0. The proof then follows using \refeq{1}, \href{https://github.com/BoltonBailey/formal-snarks-project/blob/7fd9cd122f5887f88f6a706b4f2a68a7153c7381/src/general_lemmas/mv_divisability.lean#L193}{\texttt{mul\_var\_no\_constant}} and 
\texttt{finsupp.single\_apply}, along with some \texttt{simp} lemmas.

The lemma \href{https://github.com/BoltonBailey/formal-snarks-project/blob/7fd9cd122f5887f88f6a706b4f2a68a7153c7381/src/snarks/babysnark/knowledge_soundness.lean#L381}{h4} states that :
\theoremstyle{lemma} \label{h4}
\begin{lemma}
  $B_w = b_{\gamma \beta} ZY + \sum_{i = l}^{n - 1} b'_i Y u_i(X) $
\end{lemma}
It suffices to prove that $\sum_{i = 0}^{m - 1} X^i = b_{\gamma}Z = 0$. The former claim follows from \ref{h4_1} and \ref{h2}, and the latter from \ref{h3}, along with some \texttt{simp} lemmas.

The lemma \href{https://github.com/BoltonBailey/formal-snarks-project/blob/7fd9cd122f5887f88f6a706b4f2a68a7153c7381/src/snarks/babysnark/knowledge_soundness.lean#L387}{h5} states that :
\theoremstyle{lemma}
\begin{lemma} \label{h5}
  $V_w = b_{\gamma \beta} Z + \sum_{i = l}^{n - 1} b'_i u_i(X) $
\end{lemma}
We first multiply both sides by $Y$ using \texttt{left\_cancel\_X\_mul vars.Y}. The result then follows by using \refeq{1}, \ref{h4} and \ref{h5_1}, along with some \texttt{simp} lemmas.

The lemma \href{https://github.com/BoltonBailey/formal-snarks-project/blob/7fd9cd122f5887f88f6a706b4f2a68a7153c7381/src/snarks/babysnark/knowledge_soundness.lean#L399}{h6} states that :
\theoremstyle{lemma}
\begin{lemma} \label{h6}
  $V (a_i)_{i = 0}^{l} = b_{\gamma \beta} Z + \sum_{i = 0}^{l - 1} a_i u_i(X) + \sum_{i = l}^{n - 1} b'_i u_i(X) $
\end{lemma}
This follows easily from the definition of $V$, and using \ref{h5} along with some \texttt{simp} lemmas.

The lemma \href{https://github.com/BoltonBailey/formal-snarks-project/blob/7fd9cd122f5887f88f6a706b4f2a68a7153c7381/src/snarks/babysnark/knowledge_soundness.lean#L411}{h7} states that :
\theoremstyle{lemma}
\begin{lemma} \label{h7}
  $b_{\gamma \beta} = 0$
\end{lemma}
Consider \refeq{2}, and expand on it using \ref{h6}. The statement \texttt{h6\_1} then says that the coefficient of $Z^2$ in both sides of the equation must be the same. Finally, 
using \ref{h6_2} and \ref{h6_3}, one obtains $b_{\gamma \beta}^2 = 0$, from which the result follows easily, since the field $F$ has no zero divisors. 

The lemma \href{https://github.com/BoltonBailey/formal-snarks-project/blob/7fd9cd122f5887f88f6a706b4f2a68a7153c7381/src/snarks/babysnark/knowledge_soundness.lean#L419}{h8} states that :
\theoremstyle{lemma}
\begin{lemma} \label{h8}
  $V (a_i)_{i = 0}^{l} = \sum_{i = 0}^{l - 1} a_i u_i(X) + \sum_{i = l}^{n - 1} b'_i u_i(X) $
\end{lemma}
The lemma follows easily using \ref{h6}, \ref{h7} and \texttt{simp}.

The lemma \href{https://github.com/BoltonBailey/formal-snarks-project/blob/7fd9cd122f5887f88f6a706b4f2a68a7153c7381/src/snarks/babysnark/knowledge_soundness.lean#L438}{h10} states that :
\theoremstyle{lemma}
\begin{lemma} \label{h10}
  $ (Ht + 1) \equiv (V (a_i)_{i = 0}^{l})^2 (\text{mod }t) $
\end{lemma}
Here, \texttt{singlify} has been used to think of them as single variable polynomials in terms of $X$. This follows directly from \refeq{2}.

The lemma \href{https://github.com/BoltonBailey/formal-snarks-project/blob/7fd9cd122f5887f88f6a706b4f2a68a7153c7381/src/snarks/babysnark/knowledge_soundness.lean#L449}{h12} states that :
\theoremstyle{lemma}
\begin{lemma} \label{h12}
  The multivariable constant polynomial 1 is the same as the multivariable polynomial $\sum_{n} (coeff_{1} (X^n)) X^n$, where, given a polynomial $p \in F[X]$, $coeff_{p} (X^n)$ denotes 
  the coefficient of $X^n$ in $p$.
\end{lemma}
This lemma is dependent on the definition of \texttt{polynomial.eval\_2}. The definition of this is very complicated, however, we shall remark that the proof follows directly from the lemma 
\texttt{polynomial.eval\_2\_C}.

The lemma \href{https://github.com/BoltonBailey/formal-snarks-project/blob/7fd9cd122f5887f88f6a706b4f2a68a7153c7381/src/snarks/babysnark/knowledge_soundness.lean#L454}{h13} states that :
\theoremstyle{lemma}
\begin{lemma} \label{h13}
  $ (Ht + 1) \equiv H (\text{mod }t) $ and $ (Ht + 1) \equiv 1 (\text{mod }t) $, where these polynomials are thought of as single variable polynomials.
\end{lemma}
We first apply \texttt{polynomial.div\_mod\_by\_monic\_unique}, which says that, given polynomials $f, g, q, r$, with $g$ monic, such that $f = g q + r$ and $deg(r) < deg (g)$, then,
$f \div_m g = q$ and $f (\text{mod }g) \equiv r$. Here, $\div_m$ means the quotient when $f$ is divided by $g$. We must then prove that $t$ is monic (follows from \ref{monic_t}), and 
that the degree of the constant polynomial 1 is larger than $deg(t)$ (follows from \ref{t_pos}), along with some \texttt{simp} lemmas.

We are now ready to tackle the proof of the theorem.

\subsection{Proof of main theorem}
We must prove that, given $(a_i)_{i = 0}^{l - 1}$,
$$ \sum_{i = 0}^{l - 1} a_i u_i(X) + \sum_{i = l}^{n - 1} b'_i u_i (X) \equiv 1 \texttt{mod } t $$

First, we define \href{https://github.com/BoltonBailey/formal-snarks-project/blob/7fd9cd122f5887f88f6a706b4f2a68a7153c7381/src/snarks/babysnark/knowledge_soundness.lean#L426}{\texttt{h9}} 
to be \ref{h8}, seen as an element of $F[X]$. We then use the additivity properties of the \texttt{eval\_2} map, and the lemma \ref{eval2} on \texttt{h9}. Using \texttt{h9}, it now suffices 
to prove that \newline $\text{mv\_polynomial.eval\_2 polynomial.C singlify (V a\_stmt)} ^ 2 \equiv 1 (\text{mod }t)$.

We then use \ref{h10} and \ref{eval2} to change the statement to \newline $\text{(mv\_polynomial.eval\_2 polynomial.C singlify H * t +
mv\_polynomial.eval\_2 polynomial.C singlify (mv\_polynomial.C 1))} \equiv 1 (\text{mod }t)$.

Finally, the result follows from applying \ref{h12},\ref{h13} and \ref{eval2} along with some \texttt{simp} lemmas. Yay!

\bibliographystyle{unsrt}
\bibliography{refs}

\end{document}